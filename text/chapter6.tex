\section{Schlussbetrachtung}

\subsection{Fazit und kritische Reflexion}
Die CEA ist ein von dem Analystenhaus Gartner veröffentlichtes IT-Konzept, welches darauf abzielt, Agilität, Skalierbarkeit und Anpassungsfähigkeit in Organisationen zu fördern. Diese Architektur basiert auf unabhängigen und wiederverwendbaren Komponenten, welche über APIs zu einem Gesamtsystem konsolidiert werden. Im Kontext eines Composable-ERP-Systems können somit Module wie das Finanzwesen, der Vertrieb und die Beschaffung in einzelne Services ausgelagert und nach Bedarf hinzugefügt oder entfernt werden. Um individuellen Unternehmensanforderungen gerecht zu werden, besteht die Möglichkeit, modulare Komponenten sowohl durch Eigenentwicklung als auch durch Erweiterung zu adaptieren. Um Effizienz und Anpassungsfähigkeit hierbei vollständig auszuschöpfen, ist es unerlässlich, dass diese Bausteine schnell bereitgestellt und in das bestehende System integriert werden. Dies lässt sich durch den Einsatz von CI/CD-Pipelines realisieren. Diese automatisieren den Prozess der Integration und Bereitstellung von IT-Services. Damit werden Code-Änderungen in regelmäßigen Abständen in ein gemeinsames Repository überführt, automatisch auf Fehler überprüft und in die Produktionsumgebung bereitgestellt. Das SAP DTS empfiehlt dabei eine Auswahl zwischen drei verschiedenen Tools. Dazu gehören Azure Pipelines, Jenkins und SAP CI/CD.\\ Für eine CEA ergeben sich im Vergleich zur traditionellen IT-Architektur jedoch divergierende Anforderungen an den Bereitstellungsprozess. Deshalb wurde im Rahmen dieser Arbeit ermittelt, welches CI/CD-Tool für eine CEA den größten Mehrwert birgt. Zur systematischen Evaluierung der Pipeline-Tools wurde ein Entschei\-dungs-Framework auf Grundlage des AHP-Verfahrens entworfen. In diesem Kontext erweist sich Azure Pipelines als das optimale CI/CD-Tool. Dies ist insbesondere auf den hohen Funktionsumfang des Systems zurückzuführen. So unterstützt Azure Pipelines die von der SAP veröffentlichte Programmbibliothek Project Piper, mit welcher für SAP-Technologien benötigte vorimplementierte Pipeline-Schritte ausgeliefert werden. Dies gewährleistet eine ressourcenoptimierte und standardisierte Implementierung der Bereitstellungs-Workflows. Sofern spezifische Funktionalitäten nicht durch Project Piper bereitgestellt werden, unterstützt der Azure-Pipelines-Standard darüber hinaus eine Vielzahl von Programmiersprachen, Plattformen und Test-Frameworks, welche eine maßgeschneiderte Implementierung von Pipelines ermöglichen. Zusätzlich stellt die hohe Flexibilität des CI/CD-Tools einen erheblichen Mehrwert dar. Da Azure Pipelines eine Cloud-Lösung ist, können die Rechenressourcen des Pipeline-Systems dynamisch an die spezifischen Anforderungen der Teams angepasst werden. Um eine schnelle und effiziente Bereitstellung der IT-Services zu ermöglichen, können Rechenkapazitäten während Stoßzeiten somit schnell und kosteneffektiv angepasst werden. Auch in Bezug auf die Performance erweist sich Azure Pipelines im Vergleich zu anderen CI/CD-Tools als besonders leistungsstark. Dies wird sowohl durch die Verwendung neuester Hardware-Technologien als auch durch den Einsatz von Mechanismen wie dem parallelen Ausführen von Pipeline-Schritten oder Caching erreicht.  Trotz der allgemeinen Vorteile von Azure Pipelines, sind bei der Auswahl geeigneter Bereitstellungs-Tools ebenfalls \enquote{K.O.-Kriterien} zu berücksichtigen. Da Pipelines mit Azure implementiert werden müssen, erfordert dieses Tool einen hohen Grad an DevOps-Expertise. Für Unternehmen, welche über keine oder nur begrenzte Erfahrung im Bereich CI/CD verfügen, ist von der Verwendung dieses Tools abzuraten. Stattdessen sollten diese das SAP CI/CD in Betracht ziehen. Da dieses über konfigurierbare Templates verfügt, erfordert das CI/CD-Tool bei der Pipeline-Implementierung eine deutlich geringere Expertise. Für Unternehmen, welche hingegen ein hohes Maß an Flexibilität erfordern, empfiehlt sich die Verwendung von Jenkins. Da dieses Tool in einem On-Premise-Modell betrieben wird, besitzen Unternehmen vollständige Kontrolle über das gesamte System. Dies ist insbesondere vorteilhaft für Unternehmen, welche sich in Branchen mit strikten Regularien befinden. Durch die Integration zahlreicher Plug-ins kann zum einen sichergestellt werden, dass alle in einem CI/CD-Prozess benötigten Compliance-Überprüfungen unterstützt werden. Darüber hinaus ermöglicht dies eine flexible Gestaltung der Systemsicherheit.\\ In einer kritischen Reflexion stellt sich das AHP als ein geeignetes Verfahren zur Analyse von CI/CD-Tools dar. Mithilfe dieser Methode war es möglich, dass bei der Festlegung und Gewichtung der Bewertungskriterien, Präferenzen der verschiedenen an der Bereitstellung von Software beteiligten Stakeholder berücksichtigt werden konnten. Dafür wurde ein Expertengremium aus Mitarbeitenden der SAP zusammengestellt, welche in verschiedensten Bereichen der Entwicklung und Bereitstellung von Software spezialisiert sind. Dadurch konnte ein umfassender Überblick über die Anforderungen des CI/CD-Prozesses erlangt werden. Die vorliegende Arbeit hat sich auf die Evaluation der CI/CD-Pipelines in Abhängigkeit der Technologien SAP CAP Node, SAP UI5 sowie Cloud Foundry beschränkt. Aufgrund der hohen Bedeutung der Technologieoffenheit, besteht jedoch die Möglichkeit, dass CEs divergierende Build-Tools, Test-Frameworks und Deploy- sowie Release-Funktionalitäten zur Implementierung der Pipelines benötigen. Dadurch könnte die in der vorliegenden Arbeit durchgeführte Bewertung divergierend ausfallen. Laut Experte 4, Test-Spezialist des SAP-internen CI/CD-Services, unterstützt Azure Pipelines im Vergleich zu anderen CI/CD-Lösungen eine Vielzahl von Technologien. Somit würde das Ergebnis der Bewertungen voraussichtlich ebenfalls bei der Berücksichtigung anderer Technologien ähnlich ausfallen. Folglich kann die im Rahmen dieser Arbeit durchgeführte Evaluation als angemessenes Ergebnis zur Automatisierung der Bereitstellungsprozesse einer CEA betrachtet werden \cite[Z. 59 ff.]{TestDeveloperSAPHyperspaceAdoption&Onboarding.}.

\subsection{Ausblick auf zukünftige Entwicklungen}

Obwohl sich bereits eine Vielzahl verschiedener Bereitstellungs-Tools auf dem Markt etabliert haben, besteht eine signifikante Wahrscheinlichkeit, dass CI/CD in naher Zukunft einer disruptiven Veränderung unterzogen wird. Eines dieser innovativen Bereitstellungskonzepte vermuten Forschungsinstitute wie die Linux Foundation in der KI \cite{Foundation.2022}. In diesem Kontext ergeben sich insbesondere zwei Anwendungsfelder. Dazu gehören eine \textit{Intelligente Fehlererkennung} sowie eine \textit{Generative Testerstellung}. Mit Hilfe der \textit{Intelligenten Fehlererkennung} ist es möglich, vorherzusagen, welche Code-Änderungen potenziell zu Problemen in den Produktionssystemen führen könnten. Das intelligente Generieren von Empfehlungen soll dabei zur Reduzierung des Arbeitsaufwands sowie zur Beschleunigung des Entwicklungsprozesses beitragen \cite{.20230419d}. Ein weiteres KI-Anwendungsgebiet liegt in der \textit{Generativen Testerstellung}. Durch den Einsatz intelligenter Algorithmen besteht die Möglichkeit Testfälle zu generieren, um Schwachstellen im Code aufzudecken \cite{Fernandes.20210223}. Diese umfassen neben Unit-, Integration- sowie E2E-Tests ebenfalls intelligente Security-Analysen. Damit kann während des gesamten Entwicklungsprozesses, also von der Programmierung über die Integration bis zur Bereitstellung, gewährleistet werden, dass Sicherheitslücken erkannt und potenzielle Risiken besser eingeschätzt werden.\\ Neben diesen generativen Technologien vermuten DevOps-Spezialisten ebenfalls in blockchainbasierten CI/CD-Konzepten ein hohes Potenzial \cite{.20230424}. Ein mögliches Anwendungsgebiet könnte die Versionierung von IT-Services darstellen. Dabei werden Metadaten von Code-Änderungen durch eine kryptografische Signatur gesichert und unveränderlich auf der Blockchain abgelegt. Dies erhöht die Transparenz über den gesamten Anwendungsverlauf hinweg, womit die Probleme effizienter identifiziert und Rollbacks auf frühere Versionen leichter durchgeführt werden können. Darüber hinaus kann diese unveränderliche Protokollierung ebenfalls im Rahmen der CI/CD-Validierungsprozesse eingesetzt werden. So lassen sich Testergebnisse sowie während des Code-Reviews hervorgegangene Kommentare, Bewertungen und Genehmigungen als Transaktion in der Blockchain ablegen. Damit ist es möglich, Manipulationen und Fälschungen während des Bereitstellungsprozesses zu verhindern und somit die Sicherheit der IT-Services zu erhöhen. Obwohl zahlreiche DevOps-Experten ein großes Potenzial in KI- sowie Blockchain-gestützen CI/CD-Prozessen erkennen, bleibt offen, wie sich diese Konzepte in der praktischen Anwendung bewähren. Dabei sollte etwa evaluiert werden, wie sich diese Methoden in die Bereitstellungs-Workflows der CEs integrieren lassen und ob die in dieser Arbeit evaluierten CI/CD-Tools dafür geeignet sind. Eine weiterführende Fragestellung könnte daher wie folgt lauten:\\
\textit{Inwiefern lassen sich KI- und Blockchainb-basierte CI/CD-Konzepte in die Bereitstellungsprozesse der CEs integrieren und welches Tool birgt hierfür den größten Mehrwert?}
