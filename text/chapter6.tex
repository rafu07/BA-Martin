\section{Schlussbetrachtung}

\subsection{Fazit und kritische Reflexion}
Die CEA ist ein von dem Analystenhaus Gartner veröffentlichtes IT-Konzept, welches darauf abzielt, Agilität, Skalierbarkeit und Anpassungsfähigkeit in Organisationen zu fördern. Diese Architektur basiert auf unabhängigen und wiedeverwendbaren Komponenten, welche über APIs zu einem Gesamtsystem konsolidiert werden. Im Kontext eines Composable-ERP-Systems können somit Module wie das Finanzwesen, der Vertrieb oder die Beschaffung in einzelne Services ausgelagert und nach Bedarf hinzugefügt oder entfernt werden. Um individuellen Unternehmensanforderungen gerecht zu werden, besteht die Möglichkeit, modulare Komponenten sowohl durch Eigenentwicklung als auch durch Erweiterung zu adaptieren. Um Effizienz und Anpassungsfähigkeit hierbei vollständig auszuschöpfen, ist es unerlässlich, dass diese Bausteine schnell bereitgestellt und in das bestehende System integriert werden. Dies lässt sich durch den Einsatz von CI/CD-Pipelines realisieren. Diese automatisieren den Prozess der kontinuierlichen Integration und Bereitstellung von IT-Services. Damit wird Code in regelmäßigen Abständen in ein gemeinsames Repository überführt, automatisch auf Fehler überprüft und in die Produktionsumgebung bereitgestellt. Das SAP DTS empfiehlt dabei eine Auswahl zwischen drei verschiedenen Tools. Dazu gehören Azure Pipelines, Jenkins und SAP CI/CD. Für CEAs ergeben sich im Vergleich zur traditionellen IT-Architektur jedoch divergierende Anforderungen an den Bereitstellungsprozess. Deshalb wurde im Rahmen dieser Arbeit ermittelt, welches CI/CD-Tool für CEA den größten Mehrwert birgt. Zu diesem Zweck wurde ein Entscheidungs-Framework auf Grundlage des AHP-Verfahrens entworfen, anhand welchem die CI/CD-Tools evaluiert wurden. Nach Auswertung der Analyse wurde festgestellt, dass Azure Pipelines in diesem Kontext das optimale Modell darstellt. Dies ist insbesondere auf die hohe Flexibilität und Skalierbarkeit des Tools zurückzuführen. Da Azure Pipelines eine Cloud-Lösung ist, können die Rechenressourcen des Pipeline-Systems dynamisch an die spezifischen Anforderungen der Teams angepasst werden. Um eine schnelle und effizente Bereitstellung der IT-Services zu ermöglichen, können Kapazitäten während Stoßzeiten somit kosteneffektiv angepasst werden. In Bezug auf Performance erweist sich Azure Pipelines im Vergleich zu anderen CI/CD-Tools als besonders leistungsstark. Dies wird sowohl durch die Verwendung neuester Hardware-Technologien als auch durch den Einsatz von Mechanismen wie dem parallelen Ausführen von Pipeline-Schritten oder Caching erreicht. Mit Azure Pipelines wird ebenfalls die von der SAP veröffentlichte Programmbibliothek Project Piper unterstützt. Damit werden für SAP-Technologien benötigte vorimplementierte Pipeline-Schritte ausgeliefert. Dies ermöglicht, dass CEs Bereitstellungs-Workflows ressourcenoptimiert und standardisiert implementieren können. Des Weiteren werden mit Azure Pipelines eine Vielzahl von Plattformen, Programmiersprachen und Test-Frameworks unterstützt. Dies hilft die von CEs angestrebte Technologieoffenheit aufrechtzuerhalten. Trotz der allgemeinen Vorteile von Azure Pipelines, sind bei der Auswahl geeigneter Bereitstellungs-Tools ebenfalls \enquote{K.O.-Kriterien} zu berücksichtigen. Da Pipelines mit Azure implementiert werden müssen, erfordert dieses Tool einen hohen Grad an DevOps-Expertise. Für Unternehmen, welche über keine oder nur begrenzte Erfahrung im Bereich CI/CD verfügen, ist von der Verwendung dieses Tools abzuraten. Stattdessen sollten diese das SAP CI/CD in Betracht ziehen, da dieses über konfigurierbare Templates verfügt, was bei der Implementierung der Pipelines eine deutlich geringere Expertise benötigt. Für Unternehmen, welche hingegen ein hohes Maß an Flexibilität erfordern, empfiehlt sich die Verwendung von Jenkins. Da dieses Tool in einem On-Premise-Modell betrieben wird, besitzen Unternehmen vollständige Kontrolle über das gesamte System. Dies ist insbesondere vorteilhaft für Unternehmen, welche sich in Branchen mit strikten Regularien befinden. Durch die Integration zahlreicher Plug-ins kann zum einen sichergestellt werden, dass alle in einem CI/CD-Prozess benötigten Compliance-Überprüfungen unterstützt werden. Darüber hinaus ermöglicht dies eine flexible Gestaltung der Systemsicherheit. In einer kritischen Reflexion stellt sich das AHP als ein geeignetes Verfahren zur Analyse von CI/CD-Tools dar. Mithilfe dieser Methode war es möglich, dass bei der Festlegung und Gewichtung der Bewertungskriterien, Präferenzen der verschiedenen an der Bereitstellung von Software beteiligten Stakeholder berücksichtigt werden konnten. Dafür wurde ein Expertengremium aus Mitarbeitenden der SAP zusammengestellt, welche in verschiedensten Bereichen der Entwicklung und Bereitsellung von Software spezialisiert sind. Dadurch konnte ein umfassender Überblick über die Anforderungen des CI/CD-Prozesses erlangt werden. Die vorliegende Arbeit hat sich auf die Evaluation der CI/CD-Pipelines in Abhängigkeit der Technologien SAP CAP Node, SAP UI5 sowie Cloud Foundry beschränkt. Aufgrund der hohen Bedeutung der Technologieoffenheit, besteht jedoch die Möglichkeit, dass CEs divergierende Build-Tools, Test-Frameworks und Deploy- sowie Release-Funktionalitäten zur Implementierung der Pipelines benötigen. Dadurch könnte die in der vorliegenden Arbeit durchgeführte Bewertung divergierend ausfallen. Laut Experte 4, Test-Spezialist des SAP-internen CI/CD-Services, unterstützt Azure Pipelines im Vergleich zu anderen CI/CD-Lösungen eine Vielzahl von Technologien. Somit würde das Ergebnis der Bewertungen vorrausichtlich ebenfalls bei der Berücksichtigung anderer Technologien ähnlich ausfallen. Folglich kann die im Rahmen dieser Arbeit durchgeführte Evaluation als angemessenes Ergebnis zur Automatisierung der Bereitstellungsprozesse einer CEA betrachtet werden \cite[Z. 59 ff.]{TestDeveloperSAPHyperspaceAdoption&Onboarding.}.

\subsection{Ausblick auf zukünftige Entwicklungen}

Im Verlauf der Abhandlung hat sich gezeigt, dass die untersuchten CI/CD-Tools  dazu geeignet sind, um Bereitstellungsprozesse zu automatisieren und die Qualität von IT-Services zu optimieren. Dabei besteht eine signifikante Wahrscheinlichkeit, dass CI/CD in naher Zukunft einer disruptiven Veränderung unterzogen wird. So gehen Forschungsinstitute wie die Linux Foundation davon aus, dass KI zunehmend in die Bereitstellungsprozesse integriert wird \cite{Foundation.2022}. Valerie Silverthorne, Executive Editor bei GitLab, bemerkt, dass diese Technologien das Potenzial besitzen, den Entwicklungsprozess zu revolutionieren und die Effizienz der Entwicklerteams zu erhöhen \cite{.20230415}. In diesem Kontext ergeben sich insbesondere drei Anwendungsfelder. Dazu gehören eine \textit{Intelligente Fehlererkennung}, eine \textit{Generative Testerstellung} sowie ein \textit{KI-basiertes Monitoring}. Das Konzept der \textit{Intelligenten Fehlererkennung} stellt vermutlich den vielversprechendsten Bereich innerhalb des KI-gestützen CI/CDs dar. Durch dieses Verfahren ist es möglich, vorherzusagen, welche Code-Änderungen potenziell zu Problemen in den Produktionssystemen führen könnten. Das intelligente Generieren von Empfehlungen soll dabei zur Reduzierung des Arbeitsaufwands sowie zur Beschleunigung des Entwicklungsprozesses beitragen \cite{.20230419d}. Ein weiteres KI-Anwendungsgebiet liegt in der \textit{Generativen Testerstellung}. So ist durch den Einsatz intelligenter Algorithmen möglich, Testfälle zu generieren, um Schwachstellen im Code aufzudecken \cite{Fernandes.20210223}. Diese umfassen neben Unit-, Integration- sowie E2E-Tests ebenfalls intelligente Security-Analysen. Mit diesen soll  während des gesamten Entwicklungsprozesses, also von der Programmierung über die Integration bis zur Bereitstellung, gewährleistet werden, dass Sicherheitslücken erkannt und potenzielle Risiken besser eingeschätzt werden. Das automatische Generieren von Validierungs-Workflows sorgt darüber hinaus für eine höhere Testabdeckung und erlaubt den Entwicklern, sich verstärkt auf die Implementierung neuer Funktionalitäten zu fokussieren. Durch die im \textit{KI-basierten Monitoring} abgewickelten Analysen kann der Bereitstellung-Workflow optimal überwacht und der CI/CD-Prozess somit sukzessive optimiert werden. Im Zuge dessen kann dies ebenfalls dazu verwendet werden, eine optimale Bereitstellungsstrategie, wie das Blue-Green-Deployment, Canary-Deployment oder Shadow-Deployment, zu erörtern. Somit ist es Mithilfe von KI-Modellen möglich, den Erfolg der veschiedenen Bereitstellungskonzepte vorherzusagen und somit den Softwareauslieferungsprozess zu optimieren \cite{.20230419e}. Unklar bleibt, wie sich diese Konzepte konkret in die Bereitstellungs-Workflows der CEs integrieren lassen und ob die in dieser Arbeit evaluierten CI/CD-Tools dafür geeignet sind. Eine weiterführende Fragestellung könnte daher wie folgt lauten:\\
\textit{Inwiefern lassen sich KI-basierte CI/CD-Konzepte in die Bereitstellungsprozesse der CEs integrieren und welches Tool birgt hierfür den größten Mehrwert?}
