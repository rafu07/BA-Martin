\subsection{Kodierung}
\underline{Was ist CI/CD}\\
\begin{longtable}{ |C{6cm}|C{3cm}|c|c| }
	\hline
	Aussage & Kodierung & Experte & Zeilennummer\\
	\hline
	\enquote{Dabei habe ich einen CI-Server, der mir nach einem Push in mein zentrales Repository innerhalb kurzer Zeit ein Feedback gibt.} & CI & Experte 1 & 8 ff. \\
	\hline
	\enquote{Das ist die Möglichkeit ein Feature so schnell wie möglich auf Produktion zu deployieren und für den Kunden bereitzustellen} & CD & Experte 2 & 10 ff. \\
	\hline
	\end{longtable}

\underline{Verschiedene Arten von Pipelines}\\
\begin{longtable}{ |C{6cm}|C{3cm}|c|c| }
	\hline
	Aussage & Kodierung & Experte & Zeilennummer\\
	\hline
	\enquote{[Mit der CD-Pipeline] wird zentral gebaut, getestet und gegebenenfalls auch noch Sachen wie Compliance, Vulnerabilities, statische Codechecks, Integrations-Tests und Performance-Tests abgewickelt.} & Bestandteile CD-Pipeline  & Experte 1 & 12 ff. \\
	\hline
	\enquote{Die sollte dann maximal 10 - 15 Minuten laufen. So soll der Entwickler ein schnelles Feedback bekommen.} & Pull-Request-Pipeline & Experte 1 & 26 ff. \\
	\hline
    \enquote{ Die Pipelines werden somit deutlich verkleinert, aber damit bekommt man auch schneller Feedback.} & Kleine Pipelines & Experte 2 & 29 ff. \\
	\hline
	\end{longtable}



    \underline{Deploy und Release}\\
\begin{longtable}{ |C{6cm}|C{3cm}|c|c| }
	\hline
	Aussage & Kodierung & Experte & Zeilennummer\\
	\hline
	\enquote{Das getestete Programm kann dann anschließend zum Beispiel in ein Artefakt-Repository oder in eine Produktionsumgebung bereitgestellt werden.} & Artefakt-Repository und Produktionsumgebung  & Experte 1 & 15 ff. \\
	\hline
    \enquote{Mit diesen Artefakts [im Artefakt-Repository] kann man dann auch noch sehr gut Rollbacks ausführen.} & Artefakt-Repository  & Experte 4 & 48 ff. \\
	\hline
	\enquote{Kleinen entwickelten Komponenten können mit Versionierung in das Artefakt-Repository bereitgestellt werden. Andere Entwickler können diese Komponente dann aus dem Artefakt-Repository herausziehen und für eigenen Entwicklungen wiederverwenden} & Komponentenwiederverwendung im Atrefakt-Repository & Experte 1 & 37 ff. \\
	\hline
	\end{longtable}

    
\underline{Test}\\
\begin{longtable}{ |C{6cm}|C{3cm}|c|c| }
    \hline
    Aussage & Kodierung & Experte & Zeilennummer\\
    \hline
    \enquote{Typischerweise beginnt es mit der Build-Stage, bei welcher Unit-Tests ausgeführt werden. Für CAP werden dabei die Frameworks Mocha oder Jest verwendet.} & Unit-Tests mit SAP CAP & Experte 1 & 42 ff. \\
    \hline
    \enquote{Die Integration-Tests werden mit Newman gemacht.} & Integration-Tests mit SAP CAP & Experte 1 & 46 ff. \\
    \hline
    \enquote{I.d.R. wird in der SAP Q-Unit für [Unit-Tests] verwendet} & Unit-Tests mit SAP UI5 & Experte 4 & 62 ff. \\
    \hline
    \enquote{Für Integration-Tests wird OPA5 verwendet.} & Integration-Tests mit SAP UI5 & Experte 4 & 63 ff. \\
    \hline
    \enquote{Und für System-Tests wird dann i.d.R. WDI5 verwendet.} & System-Tests mit SAP UI5 & Experte 4 & 64 ff. \\
    \hline
    \enquote{Die Empfehlung ist möglichst viel auf den unteren Ebenen abzudecken, also mit Unit-Tests und auf den oberen Ebenen nur noch das zu testen, was man nicht mit Unit- und Integration-Tests testen kann.} & Test-Pyramide & Experte 4 & 72 ff. \\
    \hline
    \end{longtable}

    \underline{Code-Analysen}\\
\begin{longtable}{ |C{6cm}|C{3cm}|c|c| }
    \hline
    Aussage & Kodierung & Experte & Zeilennummer\\
    \hline
    \enquote{[Mit SonarQube] wird dann z.B. geprüft, ob ich irgendwelche Lizenzrechte verletze.} & SonarQube & Experte 1 & 48 ff. \\
    \hline
    \end{longtable}

    \underline{Vorteile von kontinuierlicher Bereitstellung}\\
    \begin{longtable}{ |C{6cm}|C{3cm}|c|c| }
        \hline
        Aussage & Kodierung & Experte & Zeilennummer\\
        \hline
        \enquote{Wenn ich dann schnell in eine Canary-Umgebung deploye, kann ich natürlich früh Fehler finden, was schließlich auch deutlich günstiger ist.} & Frühe Fehlerfindung & Experte 1 & 22 ff. \\
        \hline
        \enquote{So ist die Gefahr, dass etwas im Produktivsystem kaputtgeht sehr gering.} & Wenig Fehler in der Produktion & Experte 2 & 25 ff. \\
        \hline
        \enquote{[Beim Feature Toggle] wird eine neue Funktionalität dann hinter einer Flag versteckt. Wenn ein bestimmter Kunde dieses Feature dann haben möchte, dann setzt er entsprechend die Flag.} & Feature Toggle & Experte 4 & 37 ff. \\
        \hline
        \end{longtable}




    \underline{CI/CD-Pipeline-Tools bei der SAP}\\
\begin{longtable}{ |C{6cm}|C{3cm}|c|c| }
    \hline
    Aussage & Kodierung & Experte & Zeilennummer\\
    \hline
    \enquote{Zum einen wird der von der SAP bereitgestellte CI/CD-Service verwendet.} & SAP BTP CI/CD & Experte 1 & 54 ff. \\
    \hline
    \hline
    \enquote{Des Weiteren gibt es Jenkins. Diese wird i.d.R. mit Project Piper verwendet.} & Experte 1 & 55 ff. \\
    \hline
    \enquote{Häufig wird für interne Projekte auch Azure DevOps verwendet.} & Experte 1 & 57 ff. \\
    \hline
    \end{longtable}

    \underline{Aspekte für Wahl einer CI/CD-Pipeline}\\
    \begin{longtable}{ |C{6cm}|C{3cm}|c|c| }
        \hline
        Aussage & Kodierung & Experte & Zeilennummer\\
        \hline
        \enquote{Für Abteilungen welche keine DevOps-Spezialisten haben spielt die Benutzerfreundlichkeit eine große Rolle. } & Benutzerfreundlichkeit & Experte 1 & 63 ff. \\
        \hline
        \hline
        \enquote{Weiterhin ist wichtig zu wissen, wie flexibel man bei der Pipeline-Gestaltung sein will.}& Flexibilität & Experte 1 & 65 ff. \\
        \hline
        \enquote{Zudem muss natürlich auch evaluiert werden, welche Funktionalitäten also Tests, Code-Scans und Builds auf der Pipeline ausgeführt werden sollen. } & Tests, Code-Analysen Build & Experte 1 & 67 ff. \\
        \hline
        \enquote{Zuletzt sollte dann was die Funktionalität angeht auch noch evaluiert werden auf welcher Plattform die Software bereitgestellt werden soll.} & Deploy und Release & Experte 1 & 68 ff. \\
        \hline
        \enquote{Skalierbarkeit spielt dann auch noch eine wichtige Rolle.} & Skalierbarkeit & Experte 1 & 69 ff. \\
        \hline
        \enquote{Integration ist auch ein sehr wichtiger Aspekt bei der Auswahl von CI/CD-Pipelines.} & Integrationsmöglichkeiten& Experte 1 & 76 ff. \\
        \hline
        \enquote{Da muss natürlich auf die Integrationsmöglichkeiten mit dem Repository geachtet werden.} & Integrationsmöglichkeiten von Repositorys & Experte 1 & 77 ff. \\
        \hline
        \enquote{Sehr selten wird eine CI/CD-Pipeline auch in die Entwicklungsumgebung integriert.} & Integrationsmöglichkeiten von Entwicklungsumgebung & Experte 1 & 81 ff. \\
        \hline
        \enquote{Wenn du in sehr großen Entwicklungen bist, dann ist es natürlich auch sehr wichtig, dass die Pipeline eine gute Performance besitzt.} & Performance & Experte 2 & 36 ff. \\
        \hline
        \end{longtable}

        \underline{SAP BTP CI/CD-Service}\\
\begin{longtable}{ |C{6cm}|C{3cm}|c|c| }
    \hline
    Aussage & Kodierung & Experte & Zeilennummer\\
    \hline
    \enquote{Was bisher noch nicht wirklich funktioniert sind API-Tests.} & Keine API-Tests &Experte 1 & 72 ff. \\
    \hline
    \enquote{Der SAP CI/CD-Service unterstützt dabei einen ganz normalen Git-Server. Was auch noch funktioniert sind BitBucket Repositorys.} & Unterstützung von Repositorys &Experte 1 & 78 ff. \\
    \hline
    \enquote{Es können dabei jedoch ausschließlich Commit Events verarbeitet.} & Unterstützung von Commit-Events &Experte 1 & 80 ff. \\
    \hline
    \enquote{ Aber ein direktes Monitoring der Pipeline gibt es nicht.} & Kein Monitoring für SAP CI/CD &Experte 1 & 82 ff. \\
    \hline
    \enquote{Eine Build-Hour kostet einen Euro.} & Kosten  &Experte 1 & 92 ff. \\
    \hline
    \enquote{ Wir können sowohl auf Cloud Foundry als auch auf Kyma deployen.} & Deployment  &Experte 1 & 85 ff. \\
    \hline
    \enquote{Nein leider nicht. Aber die Pipeline kann Software auf das Transport Management System bereitstellen.} & Parallel Build und SAP CTM  &Experte 1 & 95 ff. \\
    \hline
    \enquote{Für interne Projekte darf der SAP BTP CI/CD aufgrund der derzeitigen Produktstandards nicht verwendet werden.} & Nicht für interne Projekte  & Experte 2 & 46 ff. \\
    \hline
    \enquote{Das SAP CI/CD-Tool lohnt sich insbesondere für Kunden, die noch nicht viel DevOps-Expertise besitzen und auch keine teure Infrastruktur betreiben wollen.} & Für Kunden mit wenig Expertise  & Experte 2 & 48 ff. \\
    \hline
    \end{longtable}

       
    \underline{Azure Pipelines}\\
    \begin{longtable}{ |C{6cm}|C{3cm}|c|c| }
        \hline
        Aussage & Kodierung & Experte & Zeilennummer\\
        \hline
        \enquote{Hierbei gibt es jetzt auch einige speziellen Governace-Checks, die darüber möglich sind.} & Governance-Checks & Experte 4 & 52 ff. \\
        \hline
        \enquote{Das SAP Tools Team hatte dann alles auf einen zentralen Blick und konnte entsprechend sagen, dass wenn für eine Pipeline mehr Kapazität benötigt wird, dass entsprechend Ressourcen zugeschaltet werden.} & Erhöhte Flexibilität & Experte 4 & 53 ff. \\
        \hline
        \end{longtable}


    \underline{Security}\\
    \begin{longtable}{ |C{6cm}|C{3cm}|c|c| }
        \hline
        Aussage & Kodierung & Experte & Zeilennummer\\
        \hline
        \enquote{Früher gab es dabei immer ein Security-Experten der sich vor der Auslieferung dann um alles kümmern musste.} & Security damals & Experte 3 & 8 ff. \\
        \hline
        \enquote{Security sollte nicht mehr nur von einem Spezialisten behandelt werden. Vielmehr sollte dies als Kollektiv abgehandelt werden.} & Security heute  & Experte 3 & 32 ff. \\
        \hline
        \hline
        \enquote{Jeder muss bei der Entwicklung seiner Funktionalitäten schon so früh wie möglich schauen, ob alle sicherheitsrelevanten Aspekte eingehalten wurden. Das wird dann i.d.R. durch Automatisierung gemacht.} & Automatisierung mit Tools  & Experte 3 & 32 ff. \\
        \hline
        \enquote{[Bei statischen Codde-Analysen wird insbesondere OS-Scanning betrieben.]} & Automatisierung mit Tools  & Experte 3 & 23 ff. \\
        \hline
        \enquote{[Bei Dynamic Application Security Testing] werden dann auch tatsächlich UI-Elemente, APIs sowie Datenbanken gescannt.} & Dynamic Application Security Testing & Experte 3 & 24 ff. \\
        \hline
        \enquote{Für CAP Node wird von der SAP das Tool Checkmarx vorgeschrieben. Für Open-Source ist das Tool Whitesource vorgeschrieben.} & Security-Scans für SAP CAP Node  & Experte 3 & 36 ff. \\
        \end{longtable}


     