\section{Anwendung der Methodik auf die theoretischen Grundlagen}

\subsection{Prototypische Implementierung der Integrations- und Bereitstellungs-Pipelines}
\subsection{Evaluation der Integrations- und Bereitstellungs-Pipelines unter Anwendung des Analytischen Hierarchieprozesses}
Als Entscheidungsalternativen werden CI/CD-Pipelines für die Bereitstellung von Cloud-Software gegenübergestellt. Konkret handelt es sich hierbei, um die Tools \textit{SAP CI/CD}, \textit{Jenkins}, sowie \textit{Azure Pipelines}. Die Evaluation beschränkt sich dabei auf diese CI/CD-Pipelines, da diese zur Bereitstellung von Cloud-Software von der als SAP Best Practice definiert werden. Bei der Untersuchung der Pipelines müssen dabei verschiedene Aspekte beachtet werden. Zum einen soll die Pipeline auf Eignung zur Bereitstellung von auf Composable-Enterprise-Architektur basierender Software untersucht werden. Darüber hinaus muss evaluiert werden, wie gut sich die Tools für die Technologien SAP CAP bzw. SAP UI5 sowie für eine Bereitstellung auf der Laufzeitumgebung Cloud-Foundry der SAP Cloud-Plattform eignet. Diese Pipelines.
\subsubsection{Festlegung der AHP-Entscheidungsalternativen}
 Die zur Durchführung des AHP-Verfahrens benötigten Daten werden neben  Literaturrecherche ebenfalls mittels Experteninterviews erhoben. Zur Durchführung der Interviews wurde eine Expertengruppe aus X Mitarbeitenden der SAP zusammengestellt. Diese sind jeweils in spezifischen Bereichen der Cloud-Fullstack-Entwicklung, Test-Management, sowie DevOps spezialisiert. Diese umfassende Auswahl erlaubt es Expertise über verschiedene Fachbereiche hinweg zu gelangen und Anforderungen aller an der Entwicklung, Bereitstellung sowie Betrieb von Software beteiligten Stakeholdern zu erheben. Die in den Experteninterviews erhobenen Daten werden dabei ebenfalls zur Festlegung der Entscheidungskriterien verwendet. Dafür wurde eine induktive Kodierung der Expertengespräche vorgenommen. Aus besonders häufig von Experten genannten Aspekten systematisch Kategorien abgeleitet. Diese Kategorien wurden dabei ebenfalls als Entscheidungskategorien im AHP-Verfahren wiederverwendet. Da die CI/CD-Pipelines im Rahmen dieser Arbeit insbesondere für Kundenprojekte evaluiert werden, stellten sich die praktischen Projekterfahrungen der Experten als geeignete Datenquelle zur Festlegung der Entscheidungskriterien dar. Somit konnte evaluiert werden, aus welchen typischen Komponenten die CI/CD-Piplines in Kundenprojekten besteht bzw. welche Anforderungen an diese gestellt werden. Da innerhalb dieser Kundenprojekte oft weniger strikte Qualitätsanforderungen an den CI/CD-Prozess gestellt werden, wurde darüber hinaus erarbeitet, welche internen Bestimmungen, von der SAP zur Bereitstellung von Standardsoftware definiert werden. Obwohl diese Standards nur in wenigen Kundenprojekten Anwendung finden, ermöglicht ihre Berücksichtigung die Erfassung von Entscheidungskriterien, welche für eine optimale CI/CD-Pipeline erforderlich sind. Um eine effektive und klare Entscheidung treffen zu können, sollten im AHP-Verfahren ebenfalls Aspekte verglichen werden, bei sich die Entscheidungsalternativen besonders unterscheiden. Dafür wurde mittels Literaturanalyse sowie Konsultation von Online-Blogs Unterschiede der zu vergleichenden Pipelines erarbeitet. Falls es sich in einem spezifischen Anwendungsfall eignet, wurden aus diesen Erkenntnissen ebenfalls Entscheidungskriterien abgeleitet.\\
 Die mit dieser Vorgehensweise erarbeiteten Entscheidungsalternativen sind folgender Abbildung zu entnehmen. Auf der obersten Ebene des AHP-Entscheidungsbaums werden neun Kategorien berücksichtigt. 





\subsection{Entwicklung einer ganzheitlichen Bereitstellungsstrategie}
