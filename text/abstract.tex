\begin{center}
	\section*{Abstract}
\end{center}
Die Composable-Enterprise-Architektur (CEA) ist ein von dem Analystenhaus Gartner veröffentlichtes IT-Konzept, welches darauf abzielt, Agilität, Skalierbarkeit und Anpassungsfähigkeit in Organisationen zu fördern. Diese System-Architektur besteht aus unabhängigen, in sich geschlossenen Services, welche bei Bedarf erweitert, verändert oder ausgetauscht werden können. Um diese modularen Software-Bausteine effizient in Produktionssysteme bereitstellen zu können, bedarf es einer Continuous-Integration-and-Delivery-Pipeline (CI/CD-Pipeline). Diese automatisiert den Prozess der Integration und Bereitstellung und ermöglicht, dass Code-Änderungen zuverlässig in lauffähige Anwendungen umgesetzt werden.
Ziel der vorliegenden Arbeit ist, zu evaluieren, welches CI/CD-Pipeline-Tool zur Automatisierung der Bereitstellungsprozesse für eine CEA den größten Mehrwert birgt. Konkret werden dabei die Tools Azure Pipelines, Jenkins sowie SAP CI/CD verglichen. Zur Beantwortung der vorliegenden Forschungsfrage wurde der Analytische Hierarchieprozess (\acs{AHP}) durchgeführt. Dabei wurden die zu vergleichenden CI/CD-Tools anhand von neun gewichteten Kriterien verglichen. Die zur Durchführung des AHP-Verfahrens benötigten Daten wurden mittels semistrukturierter Leitfadeninterviews erhoben. Die Auswertung des AHP-Verfahrens veranschaulicht, dass Azure Pipelines als das optimale CI/CD-Tool angesehen werden kann. Dies ist insbesondere auf die hohe Flexibilität der Pipeline zurückzuführen. Da Azure Pipelines eine Cloud-Lösung ist, können die Rechenressourcen des Pipeline-Systems dynamisch an die spezifischen Anforderungen eines Composable-Enterprises (CEs) angepasst werden. Zudem stellt Azure Pipelines essenzielle Funktionalitäten bereit, welche für das Bauen, Testen und Bereitstellen von auf SAP-Technologien basierenden Applikationen benötigt werden. Das im AHP-Verfahren ermittelte Resultat muss jedoch für gewisse Anwendungsfälle abgegrenzt werden. Demnach wird empfohlen, dass Teams mit geringer CI/CD-Erfahrung SAP CI/CD in Betracht ziehen sollten, wohingegen Entwicklungsabteilungen, welche ein hohes Maß an Kontrolle über das Pipeline-System anstreben, die Nutzung von Jenkins erwägen sollten.  