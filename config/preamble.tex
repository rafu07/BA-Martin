\documentclass[a4paper,12pt]{article}

% Korrekte Darstellung der Texte und Trennung der Wörter
\usepackage[ngerman]{babel} % ggf. Sprache anpassen
\usepackage[utf8]{inputenc}
\usepackage{lmodern}
\usepackage{nicefrac}

%\setdefaultlanguage[spelling=new, babelshorthands=true]{german}
% Mathematik
\usepackage{mathtools}
\usepackage{amssymb}

% Anpassungen des Inhaltsverzeichnisses
\usepackage[nottoc]{tocbibind}

% Für eigene Grafiken

\usepackage{tikz} 

\usepackage{caption}
\captionsetup{
    format=hang,
    width=1\textwidth,
    justification=justified,
    singlelinecheck=false,
}

\DeclareCaptionFormat{myformat}{#1#2#3\hspace{-0.9em}}


% Tabellen
\usepackage{tabularx}

% Anführungszeichen mit \enquote{...}
\usepackage{csquotes}



% Seitenränder
\usepackage[	left=3.5cm, 
			right=2.5cm, 
			head=1.25cm, 
			bottom=2cm, 
			foot=1.25cm, 
			includefoot ]{geometry}

% Abkürzungsverzeichnis
\usepackage[printonlyused]{acronym}

% Abbildungen
\usepackage{graphicx}
\graphicspath{ {images/} }

% Fußzeile und Kopfzeile
\usepackage{fancyhdr}
\renewcommand{\headrulewidth}{0.0pt}
\renewcommand{\footrulewidth}{0.5pt}

\def\footer{
	\begin{center}
		\hfill
		\fontsize{\footerFontSize}{\footerFontSize}\selectfont
		\hfill
		\thepage
	\end{center}
}

\def\header{
	\begin{center}
		\hfill
		\fontsize{\footerFontSize}{\footerFontSize}\selectfont
		\hfill
		\textit{\textsc{\leftmark}}
	\end{center}
}

% Fußnoten
\usepackage[bottom]{footmisc}

% Zeilenabstand: 1.5
\usepackage{setspace}
\usepackage{caption}
\setstretch{1.5}

% Literaturverzeichnis
\newcommand{\position}{inline} 
\newcommand{\indextype}{numeric}
\usepackage[
backend=biber,
autocite=\position,
defernumbers=true,
style=\indextype,
useprefix=true,
%	firstinits=true -> Vorname abkürzen
]{biblatex}
\bibliography{bib/bibliography.bib}

%Float-Option H für Grafiken
\usepackage{float} 

% Darstellung von Algorithmen
\usepackage[ruled,vlined]{algorithm2e}

% Links in Output-PDF
\usepackage[hidelinks, unicode]{hyperref}
\hypersetup{pdftitle = {\thesisTitle}, pdfauthor = {\name}}

% Normale Schriftart für URLs
\renewcommand{\UrlFont}{}

% Variable für das Speichern der Seitenzahl (römisch -> arabisch -> römisch)
\newcounter{pageNumber}

% Nummerierung: 2.1, 2.2 usw.
\renewcommand{\labelenumii}{\theenumii}
\renewcommand{\theenumii}{\theenumi.\arabic{enumii}.}